%=================================
%Please do not change this environment
\documentclass{annurep}
\usepackage{graphicx}
\textheight 25.cm
\textwidth 17cm
\oddsidemargin -18pt
\evensidemargin 0pt
\topmargin -50pt
\pagestyle{empty}
%=================================


\newcommand{\beq}{\begin{eqnarray}}
\newcommand{\eeq}{\end{eqnarray}}
\newcommand{\ket}{\rangle}
\newcommand{\bra}{\langle}
\newcommand{\del}{\partial}
\newcommand{\pslash}{{p\hspace{-5pt}/}}
\newcommand{\dslash}{{\del \hspace{-5pt}/}}
\newcommand{\zslash}{{z\hspace{-5pt}/}}
\newcommand{\kslash}{{k\hspace{-6pt}/}}
\newcommand{\Thep}{\Theta^+}
\newcommand\gsim{\displaystyle\mathop{>}_{\sim}}
\newcommand\lsim{\displaystyle\mathop{<}_{\sim}}


\begin{document}

%==============================================================
%    Authors and tytle
%
\begin{center}
{\large \bf Decay of $\Thep$ in a quark model}
\vspace*{0.3cm}

%Authors
A. Hosaka$^{1}$ and B. Hosaka$^{2}$\\
%Address{1}
$^{1}${\it Research Center for Nuclear Physics (RCNP), Osaka University,
Ibaraki, Osaka 567-0047, Japan}\\
%Address{2}
$^{2}${\it Research Center for Nuclear Physics (RCNP), Osaka University,
Ibaraki, Osaka 567-0047, Japan}\\
\end{center}

\vspace*{0.5cm}
%
%==============================================================
%     Main text
%
One of the distinguished features of the pentaquark particle 
$\Theta^+$ is its very narrow width~\cite{penta}.  
In this report we study the decay of $\Thep$ in the 
non-relativistic quark model~\cite{hos}.  

In the quark model, 
the decay of the pentaquark 
occurs through the fall apart process, in which the 
five quarks dissociate into a three-quark cluster, a nucleon, 
a quark-antiquark cluster, a meson, 
without pair creation of the 
quarks
as shown in Fig.~\ref{fallapart}.  
The decay amplitude is then written as a product of the 
spectroscopic factor 
and the basic interaction matrix element. 
For the latter, we employ the standard 
meson-quark interaction of the Yukawa type:
$
{\cal L}_{\rm int} 
=
-i\ g \bar \psi \gamma_5 \Phi \psi 
\sim 
\frac{g}{2m} \chi^\dagger \vec \sigma \cdot \vec \nabla \Phi \chi
\, , 
$
where we have adopted the standard notation.   

%figure---------------------------------------------
\begin{figure}[h]
\centering
\includegraphics[width=10cm,clip]{mbb.eps}
\caption{
Meson-baron couplings involving an $mqq$ coupling. 
(a) Transition of a three-quark baryon to another three-quark baryon.
(b) A decay of pentaquark baryon into a three-quark baryon and a meson.
(c) A diagram equivalent to (b).}
\label{fallapart}
\end{figure}
%figure---------------------------------------------

The matrix element is then taken between the 
$\Thep$ in the initial state and the kaon and nucleon in the 
final state:
\beq
{\cal M}_{\Theta^+ \to K^+ n} =
-i \sqrt{2} 
\bra n_f(udd) |  
 \int d^3 x \, g \bar \psi \gamma_5 \psi
e^{-i \vec q \cdot \vec x} 
|\Thep(uudd\bar s)\ket ,
\label{calM1}
\eeq
where the initial state $\Thep$ can be expressed as 
a kaon-nucleon like state with a spectroscopic factor $a$:
$
|\Thep\ket = a |(u(1)d(2)d(3))^n (u(4)\bar s(5))^{K^+}\ket 
+ \cdots \, .
$

We have computed this matrix element for several $J^P$.  
The results are summarized in Table 1.  
From there, we see that the width of the negative 
parity $\Thep$ is too wide for the state to be regarded as 
a sharp resonance.    

For the positive parity $\Thep$, the column SF (spin-flavor) 
shows the results for the $\Thep$ configuration minimizing 
the spin-flavor interaction, where the spectroscopic factor 
is $\sqrt{5/96}$~\cite{Carlson:2003xb}.  
The column SC (spin-color) is for the result for the 
configuration minimizing the spin-color interaction, which  
has a spectroscopic factor $\sqrt{5/192}$.  
We have also shown in the column JW the result for the 
case where the Jaffe-Wilzeck type of diquark correlation 
is developed~\cite{Jaffe:2003sg}.  
In this case, the spectroscopic factor becomes 
$\sqrt{5/576}$~\cite{Carlson:2003xb}
instead of $\sqrt{5/96}$, which reduces the decay width 
by the factor 6 from the result of SF. 
Typically, the decay width of a positive parity $\Thep$ is 
of order 10 MeV.  
To get an even narrower width $\sim 1$ MeV, we need further mechanism.  


%---------------------------------------------
\begin{table}[b]
\centering
\caption{\label{widthp} \small The $KN\Thep$ coupling 
constant $g_{KN\Thep}$ and decay width (in MeV) 
of $\Thep$ for $J^P = 1/2^{\pm}$.  
}
\begin{tabular}{ c c | c c c c }
\hline
 &  & \multicolumn{4}{c}{$g_{KN\Theta}$} \\
 &  & $J^P=1/2^-$ & & $1/2^+$ \\
\hline
$\bra r^2 \ket^{1/2}$ &  $\alpha_0^2$  &  & SF  &  SC  &  JW\
$1/\sqrt{2}$ fm & 3 fm$^{-2}$ & 4.1 & 7.7 & 5.5 & 3.2\\
1 fm & 1.5 fm$^{-2}$ & 3.2 & 8.4 & 5.9 & 3.4\\
\hline
\end{tabular}
\end{table}
%---------------------------------------------


%=============================================
% In the environment {thebibliography} below, 
% DO NOT REMOVE \vspace*{-0.2cm}
%=============================================

\begin{thebibliography}{9}
%
\vspace*{-0.2cm}
\bibitem{penta}
For an overview of the recent status, see presentations at  
the international workshop PENTAQUARK04 held at SPring-8, 
Jul. 20-23 (2004):
www.rcnp.osaka-u.ac.jp/$\sim$penta04/
%
\vspace*{-0.2cm}
\bibitem{Nakano:2003qx}
T.~Nakano {\it et al.}  [LEPS Collaboration],
Phys.\ Rev.\ Lett.\  {\bf 91}, 012002 (2003)
[arXiv:hep-ex/0301020].
%
\vspace*{-0.2cm}
\bibitem{hos}
A. Hosaka, M. Oka and T. Shinozaki, 
Preprint, (2004).  
%
\vspace*{-0.2cm}
\bibitem{Carlson:2003xb}
C.~E.~Carlson, C.~D.~Carone, H.~J.~Kwee and V.~Nazaryan,
Phys.\ Rev.\ D {\bf 70}, 037501 (2004)
[arXiv:hep-ph/0312325].
%
\vspace*{-0.2cm}
\bibitem{Jaffe:2003sg}
R.~L.~Jaffe and F.~Wilczek,
Phys.\ Rev.\ Lett.\  {\bf 91}, 232003 (2003)
[arXiv:hep-ph/0307341].

\end{thebibliography}
\end{document}